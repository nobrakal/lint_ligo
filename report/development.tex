\documentclass[10pt,a4paper]{article}
\usepackage[utf8]{inputenc}
\usepackage{fullpage}
\usepackage[T1]{fontenc}
\usepackage[english]{babel}
\usepackage{hyperref}
\usepackage{minted}
\usepackage{cite}

\author{Alexandre Moine}
\title{Activity report about the development of a linter for LIGO}
\date{September 14, 2020}

\begin{document}
\maketitle

\begin{abstract}
  The Tezos Foundation has provided support for the
  development of a linter for LIGO during August and September
  2020. This document describes the main aspects of this development.
\end{abstract}

\section{Executive summary}

The development led to a first publicly available version downloadable from:
\begin{center}
  \url{https://github.com/nobrakal/lint\_ligo}
\end{center}

\paragraph{Features}

\verb|lint_ligo| is a command-line tool and a library which detect
the following source of potential issues in LIGO source code:

\begin{itemize}
\item deprecated predefined types and values ;
\item unused variables ;
\item code smells ;
\item and, in the case of a PascaLIGO file, if there is mixed usage of the two dialects.
\end{itemize}

In addition to these predefined linting rules, \verb!lint_ligo! is
also \textbf{extensible}. There are two ways for users to include new
linting rules.

\begin{itemize}
\item
  By declaring a problematic code patterns. When a user-defined code pattern is detected in
  the analyzed file, a custom message is printed to the user.

\item
  By declaring a deprecated name. When such a name is detected
  in the analyzed file, a custom message is printed to the user.
\end{itemize}

These rules are specified in a configuration file, which is an optional argument
of the binary of the linter.

\paragraph{Design and implementation}

The linter was developed to be extensible: each analysis is separated from the others
and chained at the end. Hence, adding a new analysis need only small modifications of
the actual codebase.

The tool code base is also relatively small (approximately 3000 lines) because the source code
reuses many features offered by the LIGO compiler library: the parsing of LIGO files, the different
compilation passes and tools to manage variable names and code locations.

\paragraph{Community driven}

The author collaborates with the LIGO team to design the linter.

The original proposal mentioned the introduction of a hook for a
linter in the compiler. After a proof-of-concept was developed, the
LIGO team rejected the idea as it introduced unneeded complexity for
the end-user.

Moreover, the idea of detecting PascaLIGO dialects came from a member
of the LIGO team as a valuable addition.

The author also collaborates with the community to understand the kind
of linting rules that are the most relevant. A member of the community
suggested the idea of detecting unused variables which, after a
discussion with the LIGO team, led to an integration of this feature
in the linter.

\paragraph{Next steps}

\verb|lint_ligo| can be used by the LIGO community right now.
However, some work can be done to improve the usability of the tool:
there is a need for a standard set of linting rules for each LIGO
language and the pattern matching can be improved. Furthermore, it
could be interesting to extend the detection of unused variables to
catch more cases.

% FIXME: Sur ce dernier point, il faut expliciter quelles sont les
% limitations exactes de l'outil.

\section{Details about the design and the implementation of lint-ligo}

\subsection{Structure of the project}

The client only offers a way to call the library.

The entry point of the library is \verb|lib/main.ml|. Its \verb|main|
function will, given a rules file, a contract and an entry point for
the contract:

\begin{enumerate}
\item
  Parse the given rules file.

\item
  Call the compiler library to parse the contract and compile it into
  three different targets:
  the concrete syntax tree (CST),
  the \verb|Imperative| abstract syntax tree (AST),
  and the \verb|Typed| abstract syntax tree (AST).

\item
  Run the pattern-matching and the detection of the dialect of
  PascaLIGO (if needed) on the CST.

\item
  Run the detection of deprecated variables on the
  \verb|Imperative| AST.

\item
  Run the detection of unused variables on the \verb|Typed| AST.
\end{enumerate}

Each pass is controlled by a module of the following signature:
\begin{minted}{ocaml}
  type result
  val run : ast -> result
  val format : result -> (Location.t * string) list
\end{minted}

where the formatting function allows to get a list of located
warnings, which can be printed at the end to the user.

\subsection{Patterns}
Patterns are a way to capture the shape of a piece of code. They are
generated by the following grammar:

\begin{minted}{text}
<pattern> ::=
| %<identifier>        (* A named variable *)
| %<identifier>:<type> (* A named variable with a type *)
| %_                   (* A fresh variable *)
| %_:<type>            (* A fresh typed variable *)
| %( <pattern> %)      (* a pattern in a sub-tree *)
| <word>               (* Any word *)

<type> ::= (* depends on the LIGO dialect *)
\end{minted}

More precisely, given a pattern $P$ of type $T$, and an AST $A$, the
pattern-matching engine will search a node $N$ of $A$ of type $T$ and
a substitution $\sigma$ of the variables in $P$ such that $\sigma(P)$
is equal to $N$.

They are defined as the converse of traditional patterns, which are parsed to
a variant of the original structure and then compared to this structure.
Here, the key idea is not to parse patterns but rather \emph{unparse} the original
structure and then compare the two lists of tokens.

This approach allows to not extend the syntax of the targetted language to obtain
a language of patterns (that is, including metavariables and holes), which is a non-trivial
operation. Indeed, variables can occur anywhere in the code, including for symbols and
reserved keywords, which can make the parsing very difficult.

The use of unparsed patterns induces some unusual characteristics for patterns:
\begin{itemize}
\item
  Patterns are \emph{not} parsed, thus they can correspond to invalid
  LIGO code, and no warning will be issued. Such patterns will simply
  not match anything.

\item
  Patterns are not linear, meaning that a variable can appear more
  than once in a pattern.

\item
  Variables can be typed. A typed variable will only match a node of
  the given type.
\end{itemize}

\subsubsection{About unparsed patterns}

The main module is located in \verb|lib/run_pattern.ml| which is a
functor taking an unparser (from \verb|lib/unparser/|) and producing
the engine. However, most of the work does not depend on any unparser
and is located in \verb|lib/pattern.ml|.

The linter ``unparses'' the corresponding CST to a simpler tree and
then runs the pattern-matching algorithm of Rinderknecht \& Volanschi
\cite{unparsedpatterns}.

Note that this work has needed a small modification of the ReasonLIGO
CST to include more information, see
\url{https://gitlab.com/ligolang/ligo/-/merge_requests/785}.

\subsubsection{Example}
The pattern detecting a useless test in CameLIGO:
\begin{minted}{ocaml}
if %_ then %x else %x
\end{minted}

will match the following expressions
\begin{minted}{ocaml}
if true then x else x

if (b || c) then (1 + 2) else (1 + 2)
\end{minted}

% FIXME: Est-ce que cet exemple illustre toutes les constructions du langage de patterns?
% FIXME: NON, mais il faut étendre le système de type du linter pour avoir de bons exemples.

\subsection{PascaLIGO dialects}
The analysis is located in \verb|lib/pascaligo_flavor.ml| and made on
the PascaLIGO CST. It is a fold over the CST, deducing a dialect when
possible, and raising an error when two different dialects are
detected.

Indeed, each flavor corresponds to a set of syntactic particularities.
When detecting such a particularity, the linter searches if it has already
inferred a flavor. If not, it stores the corresponding flavor. Else, it
compares the stored flavor with the new one and raises an error if they
are distinct.

An option was also be added to the compiler to choose the dialect when
decompiling to PascaLIGO, see
\url{https://gitlab.com/ligolang/ligo/-/merge_requests/791}.

\subsection{Deprecated constants}

The analysis is located in \verb|lib/unused_variables.ml| and made by
a traversal of the \verb|Imperative| AST. Indeed, all deprecated
built-ins are marked as such thanks to
\url{https://gitlab.com/ligolang/ligo/-/merge_requests/782}.
Indeed, the LIGO compiler marks deprecated built-ins as such in the
\verb|Imperative| AST.

\subsection{Unused variables}

The analysis is located in \verb|lib/deprecate.ml| and made on the
\verb|Typed| AST because the structure is much simpler to analyze than
structures from earlier passes. The AST is folded to a map linking
each variable to a boolean indicating if it was used in the code or
not (note that efforts were made to properly maintain this information
in case of a name capture).

This analysis mimics the one made by the OCaml compiler: variables
with a name starting by an underscore are ignored by the analysis.

Be aware that the analysis returns an under-approximation of the whole
set of unused variables.  For example with the program:
\begin{minted}{ocaml}
  let x = 42 in
  let y = x  in
  ()
\end{minted}

Only the variable \verb|y| will be marked as unused because
\verb|x| is used in the definition of \verb|y|.

\bibliographystyle{plain}
\bibliography{publications}{}
\end{document}
