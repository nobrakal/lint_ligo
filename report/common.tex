I was funded by the Tezos Foundation to develop a linter for the LIGO language during August and September 2020.\\
The resulting project is publicly available at \url{https://github.com/nobrakal/lint\_ligo}.
It consists of a library implementing the linter, and a command-line client.
The library heavily reuses the interface of the LIGO compiler to compile partially the analyzed file.\\
The linter is modular and contains at this time four components:

\begin{itemize}
\item The detection of deprecated constants.
\item The detection of unused variables.
\item The detection of user-defined patterns.
\item In the case of a PascaLIGO file, the detection of the used dialect, and if it is mixed with the other dialect.
\end{itemize}

Each component may emit several warnings, printed to the user at the end.\\
The linter can be configured by the user through a configuration file containing two kind of rules (see the \verb|README.md| file for more information):
\begin{itemize}
\item Code patterns. When detected in the analyzed file, a custom message is printed to the user.
\item Custom deprecated name. When such a name is detected in the analyzed file, a custom message is printed to the user.
\end{itemize}

The original proposal mentioned the introduction of a hook for a linter in the compiler. After a proof-of-concept was developed, the development team of LIGO rejected the idea as it introduced unneeded complexity for the end-user.\\
On the contrary, the idea of detecting unused variables came for the LIGO community and the idea of detecting PascaLIGO dialects from the LIGO development team.