\documentclass[10pt,a4paper]{article}
\usepackage[utf8]{inputenc}
\usepackage[T1]{fontenc}
\usepackage[english]{babel}
\usepackage{hyperref}

\usepackage{minted}
\author{Alexandre Moine}
\title{Roadmap to lint\_ligo 2.0}
\begin{document}
\maketitle

This document lists several possibilities for the future of \verb|lint_ligo|\footnote{\url{https://github.com/nobrakal/lint_ligo}}, a linter for LIGO\footnote{\url{https://ligolang.org/}}.

\paragraph{Integration in integrated development environments}
A linter is even more useful when integrated into an IDE, as linting errors could be displayed directly to the user.
An integrated \verb|lint_ligo| could be a serious argument for the adoption of the linter by the community.

\paragraph{A standard set of linting rules}

The linter provides an efficient pattern-matching algorithm but it needs a concrete set of patterns to be useful. Some are available in the \verb|examples/| directory but they are too few. Code patterns should be added to create a concrete standard set of linting rules.

\paragraph{Sementic patterns}
Some code patterns target more the code semantics than its exact shape. For example, the 3 definitions of the following \verb|option| function in CameLIGO:
\begin{minted}{ocaml}
let option (default : int) (x : int option) : int =
  match x with
  | None -> default
  | Some x -> x

let option (default : int) (x : int option) : int =
  match x with
  | Some x -> x
  | None -> default

let option (default : int) (x : int option) : int =
  match x with
    None -> default
  | Some x -> x
\end{minted}

have all the same semantics and should be in some cases matched by the same ``semantic'' pattern.\\
This is actually not the case as the actual pattern-matching algorithm is based on the CST, the first structure of the compilation pipeline. This structure still contains many textual information mainly irrelevant for a ``semantic'' pattern (like the order of a pattern matching or the leading vertical bar).

An improvement could be to normalize the CST by a compilation followed by a decompilation before running the pattern-matching algorithm. Irrelevant information is erased during the first stages of the compilation and the decompilation allows to run the existing algorithm.

\paragraph{A better detection of unused variables}

The actual algorithm used for detecting unused variables returns an under-approximation of the full set of unused variables. Indeed, it only detects if each variable is being used, but does not propagate the information.\\
For example with the program:
\begin{minted}{ocaml}
  let x = 42 in
  let y = x  in
  ()
\end{minted}
Only the variable \verb|y| will be marked as unused because of \verb|x| being used in the definition of \verb|y|.

A possible improvement is to use a refined algorithm ``propagating'' the information: in the previous example, the algorithm should be aware that, as \verb|y| is unused and \verb|x| is only used in its definition, \verb|x| is finally not used.

\paragraph{Detection of unused types}

For now, the unused variables algorithm ignores types definition. It could be interesting to verify if locally defined types are actually used in the code.

\end{document}
%%% Local Variables:
%%% mode: latex
%%% TeX-master: t
%%% End:
